\documentclass{article}
\usepackage{cite}
\begin{document}
\title{Sample Document with Citations}
\author{}
\date{}
\maketitle
\section{Emerging Powers in International Politics}
he 21st century is marked by an increased attention to the appeal and positive image of a
country as instruments of influence in the international arena\cite{bohomolov2012ghost}. There has
appeared the concept of soft power, whose author, U.S\cite{sergunin2015understanding}. political
scientist Joseph Nye described it as “the ability to get what you want through attraction rather than
coercion or payments\cite{hill2006moscow}.” A nation’s image secures attractiveness and trust in a
country, playing a crucial role as the key soft power component\cite{kiseleva2015russia}. Therefore,
the efforts of states along this line relate not so much to the sphere of culture and information as to
geopolitics\cite{kosachev2012spsecific}.
\section{Atomic Force Microscopy, a Powerful Tool in Microbiology}
Understanding the functions of microbial cell surfaces requires knowledge of their structural and
physical properties\cite{dufrene2002atomic}. Electron microscopy has long been recognized as a key
technique in microbiology to elucidate cell surface ultra structure\cite{engel1999atomic}. An exciting
achievement has been the development of cryotechniques which allow high-resolution imaging of
cell structures in conditions close to the native state\cite{franz2008atomic}. Yet direct observation in
aqueous solution remained impossible.Because of the small size of microorganisms, the physical
properties of their surfaces have been difficult to study\cite{marrese2017atomic}. Quantitative and
qualitative information on physical properties can be obtained by electron microscopy techniques,
X-ray photoelectron spectroscopy, infrared spectroscopy, contact angle, and electrophoretic mobility
measurements\cite{altman2015noncontact}.
\bibliographystyle{plain}
\bibliography{references}
\end{document}